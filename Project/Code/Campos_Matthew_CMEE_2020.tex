\documentclass[11pt]{article}
\usepackage{graphicx}
\usepackage[utf8]{inputenc}
\usepackage{amsmath}
\usepackage{gensymb}
\usepackage{setspace}
\usepackage{caption}
\usepackage{subcaption}
\usepackage{wrapfig}
\usepackage{caption}
\usepackage{csvsimple}
\usepackage[font=scriptsize,labelfont=bf]{caption}
\usepackage{csvsimple}
\usepackage{amssymb}
\usepackage[title]{appendix}
\usepackage{textgreek}
\usepackage{amsmath}
\usepackage{harvard}
\usepackage{parskip}
\usepackage[a4paper,width=160mm,top=20mm,bottom=20mm,bindingoffset=6mm]{geometry}
\usepackage{fancyhdr}
\pagestyle{fancy}
\fancyhead{}


\linespread{1.25}

\begin{document}

\begin{titlepage}
    \begin{center}
    \vspace*{1cm}

    \Huge
    \textbf{\underline{Effects of Gene Flow}}\\
    \textbf{\underline{on Genetic Architecture}}\\

    \vspace*{0.5cm}
    \LARGE
    Computing Miniproject\\
    MSc. Computational Methods in Ecology and Evolution\\

    \vspace*{1.5cm}

    \large
    \textbf{Matthew Campos}\\
    \textbf{CID: 01749686}\\
    \textbf{matthew.campos19@ic.ac.uk}\\

    \vspace*{0.8cm}

    \large
    \textbf{Supervisor: Dr. Scott Rifkin}\\
    \scriptsize
    Professor of Ecology, Behavior and Evolution\\
    Division of Biological Sciences\\
    University of California San Diego\\
    sarifkin@ucsd.edu\\

    \large
    \textbf{Internal Supervisor: Dr. Thomas Bell}\\
    \scriptsize
    Professor of Microbial Ecology\\
    Department of Life Sciences\\
    Imperial College London\\
    thomas.bell@imperial.ac.uk\\

    \end{center}
\end{titlepage}


\newpage

\section{Declaration}
All raw data collected were from the simulation I created for my project. The simulation requires the input of genetic model systems, and mathematical equations used to derive output. The model system and equations used in the simulation were sourced from the work of \textit{Omholt et al,} and \textit{Gjuvsland et al}. I was reponsible for data processing, cleaning and analysis. All analyses presented in the paper are from the simulations, with the help of my supervisor.

\section{Acknowledgements}
I would like to firstly thank Dr. Scott Rifkin for being a wonderful supervisor and guiding me throughout the project. This includes understanding background knowledge, results and overall research purposes. Secondly, thank you to Dr. Thomas Bell for agreeing to be my internal supervisor, making sure I am aware of the process of the project and ensuring my safety during such difficult times. Also to Dr. Samraat Pawar and Dr. James Rosindell for co-leading a challenging but rewarding CMEE course.
\\Finally I would like to thank the laboratory of Dr. Rifkin- Antonia Darragh, Jessica Bloom, Alexis Cugini, Yang Bing and Rachel Goodridge for being very welcoming and having wonderful and insightful weekly meetings. Good luck with everything!

\newpage

\section{Abstract}

\newpage

\section{Introduction}
% ITEX root = ../thesis.tex
Species migration can result in the following: (i) it allows individuals and species to colonise new areas and create new subpopulations. Over time, ecological events cause species to become reproductively isolated. This is known as allopatric speciation and it leads to the splitting of lineages, differentiating both related populations long-term. This differentiation also increases with geographic distance. Without gene flow, both sets of species are able to rapidly evolve in their local optimums /cite{garcia1997genetic}. Overall, species that are reproductively isolated have more pronounced modifications which can be observed phenotypically and genotypically \cite{pongratz2002genetic,sato2006effect}. (ii) Migration can allow isolated species to attempt to colonise each other’s habitats. If species are capable of interbreeding, parapatric or peripatric speciation may occur, depending on distance. This introduces new sets of alleles into an environment and as species interact and pass on its heritable genes, it changes the developing genetic makeup of local species. The latter case includes gene flow which helps to maintain the genetic diversity in an area but has been shown to homogenize populations over long periods of time, through the recombination of genes \cite{sato2006effect}. Advancements in genotypic techniques now enable us to study the genotypic effects and further our understanding of phenotype-genotype relationship. As organisms evolve, phenotypic evolution is assisted with genotypic evolution. Collective expression of certain genes through pathways assist in the morphology and behaviour we observe in species. Hereditary genome alterations through random changes in molecular mechanisms change varying aspects of the species \cite{chandrasekaran2008origins} These molecular changes induced by mutation and recombination lead to the variation of descending species \cite{chandrasekaran2008origins,ohno1999gene,brown2002genomes}. Over time, evolutionary forces involving drift and selection acts on these polymorphisms and those most fit passes their variant genome, and phenotype as a result to future generations. This is the foundation of Darwin’s theory natural selection.
\\To better understand species evolution, we can focus on the development of their genome network. Orr showed that there is variation with respect to genetic differences or gene influence on phenotype. The effects of adaptive and non-adaptive processes vary among species where there is no common set of genes involved, nor is the effects and interactions of the genes similar for species \cite{orr1998population} Although generalizations cannot be made of genetic function and interactions, what can be considered is the pattern at which these genetic processes develop over time. Genetic network simulations can be used to understand these patterns of evolution and the effect on phenotype-genotype relationships. Long temporal periods allow genetic interactions within a network to robustly develop, canalising the network \cite{orr1998population,lynch2007evolution}. Lynch highlighted the significance of non-adaptive processes as well in shaping genetic networks. His study showed that networks can still evolve its architecture and become redundant even without the influence of natural selection \cite{lynch2007evolution}. Robustness can evolve from the effects of epistasis, additivity and dominance, all of which are connected \cite{omholt2000gene}.
\\Species evolution is non-linear, descending with modification and constant splitting from lineages of a common ancestor. This continuous process over long temporal periods results in the accumulation of optimal genetic adaptations that results in a robust network structure. This can be quantified through fitness or reproductive success. There is a balancing act as selection aids to propagate fitter variants in a population, while mutation and environmental change limits such propagation \cite{burt1995evolution}. What this study focuses on is how these forces affect the development of genes and the genetic network. Specifically, with the effects of gene flow on a genetic network that has evolved in isolation. The patterns of change that a genetic network undergoes with these adaptive and non-adaptive evolution processes. Even once a robust structure is reached, how does the structure resist change and maintain its network despite perturbations and evolutionary processes. As species evolve, studies have shown that pathways have a safety margin, that make them resistant to change such as mutations \cite{bourguet1999evolution}. Species best suited to their environment will evolve to their local optima, which we can represent as a quantitative value. The further apart these values are, what I label as environmental distance, the greater the variance of the two species. The concern is on how a network responds when these evolutionary forces come into play and seeing the evolving genetic interactions. Investigating the effects of changing migration rates, two variant genetic networks, environmental distance and patterns of migration.
\\Ecological events eliminate barriers and allow species to migrate into new environments, introducing new sets of genes in an environment. The presence of variant genes and network structures from gene flow hinders local adaptation and fixation of adaptive genes \cite{burt1995evolution}. Using quantitative trait loci (QTL) we are able to numerically interpret and visualise the patterns of change. Previous research looked at the effects of gene flow, selection and mutation at generating local adaptation at the phenotypic level, showing how maintenance of alleles and linkage is important in adaptation \cite{yeaman2011genetic}. It was shown that with random perturbations and aid of genetic modifiers, there are bounds for which selection for canalization can act on, leading to evolution of robustness. They also showed that under migration selection balance, selection for robustness increases with the migration rates \cite{proulx2005opportunity}. This research will be looking at the changes in genetic architecture dynamics and the interactions of the varying systems. As a genetic network evolves, there exists a threshold which is actively regulating these homeostatic genes \cite{gjuvsland2007threshold}. As selection for robustness occurs within the local population, it can give insight into the change in architecture and statistically significant interaction \cite{gjuvsland2007statistical}.
\\Using a multi-locus system, I will construct a genetic network and simulate the effects over many generations and see how the output of the network changes, specifically looking at allelic interactions and tracking the fitness over time. Variance in fitness should decrease as a genetic network becomes robust, making it resistant to perturbations. Fitness can be quantified as reproductive success and is represented by passing on quantitative values generated from the alleles. These values are used to derive the trait values of individuals of which phenotypic values are then calculated and used as probabilities for fitness. The expectation is that after migration, a more robust network is formed when compared to before migration. At the start allowing new alleles to enter the population will result in a less robust network and more susceptible to perturbations \cite{garcia1997genetic}. Especially when the migrant network is a different structure, gene flow will allow maladaptive alleles to enter and should those be passed on, will impose a fitness cost to individuals \cite{tigano2016genomics} However, over time, the network should adapt and become resistant to such perturbations.


\newpage

\section{Methods}
% ITEX root = ../Campos_Matthew_CMEE_2020.tex
I wrote a R script that constructs a genetic network, and a variant form, and simulates its evolutions, allowing migration to occur between two populations. All functions to perform adaptive and non-adaptive processes were written from scratch and implemented in the simulation. The following functions are:
\begin{itemize}
    \item Population: initialises the starting populations of specified size where each individual (row) contains 12 allele sites (4 per gene). Since it is a di-allelic model, it is a 2-dimensional array.
    \item Fitness: determines the fitness value of each individual based on their trait values and used as a probability for offspring contribution. A heavy tailed Cauchy distribution is used to determine fitness value from trait values. Each individual has a probability of passing on their genotype to the next generation and function randomly samples from the distribution to select parents, representative of genetic drift.
    \item Mutation: produces an array same dimensions as the population and random uniformly distribution of values to determine which sites undergo mutation based on inputted mutation rate. Generates a new value using a normal function with current value as the mean and a standard deviation of 0.001.
    \item Recombination: randomly chooses which site, and if any consecutive sites downstream, to switch allele values for each individual.
    \item Migration: using a uniform distribution, randomly generates values for each individual in the migrant population to determine which individuals will migrate and replace those in the main population. Population is kept constant in both populations, representing a balanced dispersal between immigration and emigration \cite{rice2009evolution,w2004dispersal}.
\end{itemize}
\subsection{The model}
A di-allelic interlocus model from the research of \textit{Omholt et al}(2000). In this case, all the genes are hereditary, representing only the regulatory and coding region which determine protein expression and rate of expression. Studies has shown that mutations along the coding region are known to cause morphological variation within species \cite{stern2009genetic}. This model structure evolves dominance through epistatic interactions and autoregulatory effects. Using a system of equilibrium solutions and solved ordinary differential equations (ODE), simulated protein concentrations corresponding to phenotype are measured over time \cite{omholt2000gene}. Here I consider the loci as quantitative factors of protein function, and trait value is determined by protein concentrations. The greater the amount of protein expressed, the larger the trait value. The model consists of three genes, $X_1$, $X_2$ and $X_3$. Let \(j\) represent the genes where $j = {1,2,3}$, each gene $X_j$ consists of two alleles, $X_{j1}$ and $X_{j2}$. This leads to the formula:
\begin{equation*}
    y_j = X_{j1}  + X_{j2} \label{eq:Protein Expression} \tag{1}
\end{equation*}
\begin{wrapfigure}{l} {0.7\textwidth}
    \begin{center}
        \includegraphics[scale=0.35]{../Results/Model_diagram.jpg}
    \end{center}
    \caption{Diagram showing the genetic model and two variants used to represent the migrant population. (a) Interlocus model of the population in focus. Lines labelled with mathematical symbols showing the interactions between genes. Gene $X_1$ interacts with both gene $X_2$ and $X_3$, positively regulating both of them. To limit site values below infinity, gene $X_2$ is reponsible for negatively autoregulating $X_1$. There is an output for each gene where $j = {1,2}$ and $y_j = x_{j1} + x_{j2}$. Gene $X_3$ contains the trait values for each individual, which is the output. Circle represents phenotype which is determined from trait values using a Cauchy Distribution. (b) and (c) represent models for the migrant population. (b) is the same pathway and regulation as (a) however (c) is switched where gene $X_1$ negatively autoregulates $X_2$, and gene $X_2$ positively regulates $X_1$ and $X_3$. Again, output of gene $X_3$ are the trait values used to derive fitness.}
    \label{fig:Starting parameters}
\end{wrapfigure}
Where $y_j$ is the total protein concentration at each gene. There are four sites which represent the different factors affecting protein production. These are \textit{\textalpha}, \textit{\textgamma}, \textit{\texttheta} and $P$. \textit{\textalpha} is the protein production rate while \textit{\textgamma} is the degradation rate \cite{omholt2000gene}. For both sets of populations, a single gene, $X_3$ determines the trait value for individuals and quantifiably differentiates the populations in terms of morphology \cite{orr2001genetics}. For the population in focus, gene $X_1$ positively regulates gene $X_2$ and gene $X_3$, and gene $X_1$ is negatively regulated by gene $X_2$. This is to regulate trait value and prevent the value from exceeding to infinity. As gene $X_2$ increases in expression, it decreases $X_1$ expression, negatively autoregulating the system and limiting its value. Let $j = {1,2,3}$ and $i = {1,2}$, from the separate researches of \textit{Omholt et al}(2000), and \textit{Gjuvsland et al}(2007), $R_{j}$ is a regulatory Hill Function representing a Michaelis-Menten mechanism, where $S(y_j,\theta,P) = \frac{y_j^P}{y_j^P+\theta^P}$. The Hill Function explains the relationship between regulator and producer, where \texttheta is the amount of regulator needed for 50\% production rate and P affects the steepness of the curve \cite{gjuvsland2007statistical,omholt2000gene}. Should the network be negatively regulated, it leads to the following equation:
\begin{equation*}
    R_{j}(y) = 1 - S(y, \theta_j , P_j), j = {1, 2} \label{eq:Negative autoregulation function} \tag{2}
\end{equation*}
And if positively regulated:
\begin{equation*}
	R_{j}(y) = S(y, \theta_j, P_j), j = {1, 2} \label{eq:Positive autoregulation function} \tag{3}
\end{equation*}
Again, letting $j = {1,2,3}$, as gene $X_1$ positively autoregulates gene $X_2$ and gene $X_3$, and gene $X_2$ negatively autoregulates gene $X_1$, this results in the following equations:
\begin{equation*}
    R_{1j}(y_2) = 1 – S(y_2, \theta_{2j}, P_{2j}) \label{eq:X1 negative autoregulation function} \tag{4.1},
\end{equation*}
\begin{equation*}
    R_{2j}(y_1) = 1 – S(y_1, \theta_{1j}, P_{1j}) \label{eq:X2 positive autoregulation function} \tag{4.2},
\end{equation*}
\begin{equation*}
    R_{2j}(y_1) = 1 – S(y_1, \theta_{3j}, P_{3j}) \label{eq:X3 positive autoregulation function} \tag{4.3}
\end{equation*}
\textit{\textmu} is the ratio of \textalpha and \textgamma per locus. Using the equilibrium solutions, total protein concentration is calculated by the following equations:
\begin{equation*}
    y_1 = \mu_{11}(1 – S(y_2, \theta_{21}, P_{21})) + \mu_{12}(1 – S(y_{2}, \theta_{22}, P_{22})) \label{eq:y1 function} \tag{5.1}
\end{equation*}
\begin{equation*}
    y_2 = \mu_{21}(S(y_{1}, \theta_{11}, P_{11})) + \mu_{22}(S(y_{1}, \theta_{12}, P_{12})) \label{eq:y2 function} \tag{5.2}
\end{equation*}
\begin{equation*}
    y_3 = \mu_{31}(S(y_{1}, \theta_{31}, P_{31})) + \mu_{32}(S(y_{1}, \theta_{32}, P_{32})) \label{eq:y3 function} \tag{5.3}
\end{equation*}
\subsection{Migrant network}
For the first motif, the genetic network will be the same as the main population, just evolving to a different local optimum trait value of either 65 or 80. For the second motif however, the difference is that gene $X_1$ negatively regulates gene $X_2$, while gene $X_3$ and gene $X_1$ are positively regulated by gene $X_2$. The formulas used to derive $y_1$, $y_2$ and $y_3$ values for the migrant population are as follows:
\begin{equation*}
    y_1 = \mu_{11}(S(y_2, \theta_{21}, P_{21})) + \mu_{12}(S(y_2, \theta_{22}, P_{22})) \label{eq:y1 function} \tag{6.1}
\end{equation*}
\begin{equation*}
    y_2 = \mu_{21}(1 – S(y_1, \theta_{11}, P_{11})) + \mu_{22}(1 – S(y_1, \theta_{12}, P_{22})) \label{eq:y2 function} \tag{6.2}
\end{equation*}
\begin{equation*}
    y_3 = \mu_{31}(S(y_2, \theta_{31}, P_{31})) + \mu_{32}(S(y_2, \theta_{32}, P_{32})) \label{eq:y3 function} \tag{6.3}
\end{equation*}
This is to represent the concept of differentiated species but can still integrate in the other population and interbreed.
\subsection{The simulation}
A total of 44 permutations based on conditions in \textit{(see Appendix A)} of environmental distance, genetic network structure, migration rates and migration patterns were simulated for 1,200 generations each run. For the effect of genetic drift and to account for the large deviations of values, a Cauchy distribution is used to generate fitness probabilities per generation. Since the Cauchy distribution is characterized for its heavy tails. The values entered in the Cauchy distribution are the desired trait values. It is important to note that environment is kept constant. Both populations were kept constant at 500 individuals. The main population evolved to a trait value of 50 with a standard deviation 8, while the migrant population alternated between 65 and 80 with standard deviation 10. The large standard deviations characterise the varying forms of morphology that can be noticed in species. The trait values represent the environments of both populations and the local optimums they evolve to.\\
For the simulation we assume that both populations have the same size and stay constant, with migrants replacing individuals.  There is no spatial structure and all individuals have an equal chance of being replaced. Both populations undergo divergent selection, stabilising in their own environments to different specified trait values, thus differentiating the populations over time \cite{sato2006effect}. Alleles for each individual can either be homogenous, using a uniform distribution to determine starting value between 0.1 and 0.3 for both populations, or heterogenous, using a uniform to randomly generate the starting allele values, again between 0.1 and 0.3. If the population is homogenous, each individual in the population starts with the same value at each locus, otherwise have differing values if heterogenous.
Recombination is equal chance at any locus and interchanges the alleles and everything downstream. Mutation can occur at each locus by randomly deviating from the current value. The probability is the same constant for both populations where each locus has an equal chance of mutating. Mutation probability is kept constant at 0.0011 per site. A mutation in the second gene will have trans-regulatory effects as gene $y_2$ negatively autoregulates gene $y_1$, while the effect of gene $y_1$ will affect the expression values of gene $y_3$. Since all genes in the model represent regulatory and coding regions, mutations in any site can be considered to affect phenotype, for its pleiotropic effects \cite{rice2019evolution,landry2007genetic}.\\
Fitness is reproductive success, or the probability of being a parent and passing on their allele values which is determined by phenotypic value. Each individual per generation has no limit as to how many times they can be a parent, however the standard deviation of 8 and 10 in the Cauchy distribution attempts to produce varying combination of parents. Migration rates varied between 1\%, 3\% and 5\%. As migrant individuals enter the population, they randomly replace individuals in the population. With constant population size, this represents immigration and emigration. Furthermore, low migration rates were used to prevent migration population from completely replacing the original population and allowing the network to be able to adapt to the new values. Both populations have a burn-in period of 80 generations to evolve in their own environments before migration can happen. Also, migration only occurs till the 700th generation. The remaining 500 generations are to assess how the network responds to the migration. Patterns of migration were also considered, varying between each generation, every 10 generations, every 5 generations and random (between 1\% and 5\% each occurrence) after the 80th generation.
\subsection{Analysis}
Analysis was done on the recorded fitness, trait values and population arrays. At the end of each simulation, fitness is normalised by dividing fitness probabilities with the median of Cauchy distribution for the local population, with 1.0 being the highest possible fitness value. Firstly, control conditions of no migration were simulated to see how rapidly isolated networks evolve. Without migration, I expect rapid evolution of allele values, especially for the heterogenous population due to varying alleles present \cite{garcia1997genetic}. To analyse robustness, I calculate a robustness ratio. The fittest individuals before and 10 generations after migration were recorded and replicated such that there were 4 separate mutations per site. Trait values were again inputted into the Cauchy distribution to determine fitness values. The robustness ratio is then the variance in fitness after migration divided by variance in fitness before migration. Ratios were then log transformed as to linearise and make it less skewed. A negative value is thus desired for robustness and analysis of variance test is done to see if any factors significantly contribute to robustness.


\section{Results}


\section{Discussion}


\section{Concluding remarks and looking forward}


\newpage

\section{Data Code and Availability}
All code can be found in the GitHub Repository:\\
{https://github.com/matthewcampos/CMEECourseWork.git}

\newpage

\bibliographystyle{agsm}
\bibliography{references}

\newpage

\begin{appendices}
\section{Permuations of Conditions}
\textbf{Description:} The different combinations of conditions used for data sorting and simulations. Data folders are permutations of Genetic structure and Environmental distance i.e. Same and 30 (Main population 50 and Migrant population 80). Within each folder are 44 combinations of starting genetic makeup of populations, migration rate and migration pattern for the simulations. For example, within Same and 30 is: homogenous - heterogenous, 3%, every generation.
\begin{table}[h]
\centering
\includegraphics[scale=0.40]{../Results/AppendixI_conditions.jpg}
\end{table}

\newpage

\section{Computer Program Workflow}
\textbf{Description:} Workflow of R script of simulation program and the logic behind the design of the simulation.
\begin{figure}[h]
\centering
    \includegraphics[width=0.9\textwidth]{../Results/workflow.jpg}
\end{figure}

\end{appendices}

\end{document}
