% ITEX root = ../thesis.tex
Species migration can result in the following: (i) it allows individuals and species to colonise new areas and create new subpopulations. Over time, ecological events cause species to become reproductively isolated. This is known as allopatric speciation and it leads to the splitting of lineages, differentiating both related populations long-term. This differentiation also increases with geographic distance. Without gene flow, both sets of species are able to rapidly evolve in their local optimums /cite{garcia1997genetic}. Overall, species that are reproductively isolated have more pronounced modifications which can be observed phenotypically and genotypically \cite{pongratz2002genetic,sato2006effect}. (ii) Migration can allow isolated species to attempt to colonise each other’s habitats. If species are capable of interbreeding, parapatric or peripatric speciation may occur, depending on distance. This introduces new sets of alleles into an environment and as species interact and pass on its heritable genes, it changes the developing genetic makeup of local species. The latter case includes gene flow which helps to maintain the genetic diversity in an area but has been shown to homogenize populations over long periods of time, through the recombination of genes \cite{sato2006effect}. Advancements in genotypic techniques now enable us to study the genotypic effects and further our understanding of phenotype-genotype relationship. As organisms evolve, phenotypic evolution is assisted with genotypic evolution. Collective expression of certain genes through pathways assist in the morphology and behaviour we observe in species. Hereditary genome alterations through random changes in molecular mechanisms change varying aspects of the species \cite{chandrasekaran2008origins} These molecular changes induced by mutation and recombination lead to the variation of descending species \cite{chandrasekaran2008origins,ohno1999gene,brown2002genomes}. Over time, evolutionary forces involving drift and selection acts on these polymorphisms and those most fit passes their variant genome, and phenotype as a result to future generations. This is the foundation of Darwin’s theory natural selection.
\\To better understand species evolution, we can focus on the development of their genome network. Orr showed that there is variation with respect to genetic differences or gene influence on phenotype. The effects of adaptive and non-adaptive processes vary among species where there is no common set of genes involved, nor is the effects and interactions of the genes similar for species \cite{orr1998population} Although generalizations cannot be made of genetic function and interactions, what can be considered is the pattern at which these genetic processes develop over time. Genetic network simulations can be used to understand these patterns of evolution and the effect on phenotype-genotype relationships. Long temporal periods allow genetic interactions within a network to robustly develop, canalising the network \cite{orr1998population,lynch2007evolution}. Lynch highlighted the significance of non-adaptive processes as well in shaping genetic networks. His study showed that networks can still evolve its architecture and become redundant even without the influence of natural selection \cite{lynch2007evolution}. Robustness can evolve from the effects of epistasis, additivity and dominance, all of which are connected \cite{omholt2000gene}.
\\Species evolution is non-linear, descending with modification and constant splitting from lineages of a common ancestor. This continuous process over long temporal periods results in the accumulation of optimal genetic adaptations that results in a robust network structure. This can be quantified through fitness or reproductive success. There is a balancing act as selection aids to propagate fitter variants in a population, while mutation and environmental change limits such propagation \cite{burt1995evolution}. What this study focuses on is how these forces affect the development of genes and the genetic network. Specifically, with the effects of gene flow on a genetic network that has evolved in isolation. The patterns of change that a genetic network undergoes with these adaptive and non-adaptive evolution processes. Even once a robust structure is reached, how does the structure resist change and maintain its network despite perturbations and evolutionary processes. As species evolve, studies have shown that pathways have a safety margin, that make them resistant to change such as mutations \cite{bourguet1999evolution}. Species best suited to their environment will evolve to their local optima, which we can represent as a quantitative value. The further apart these values are, what I label as environmental distance, the greater the variance of the two species. The concern is on how a network responds when these evolutionary forces come into play and seeing the evolving genetic interactions. Investigating the effects of changing migration rates, two variant genetic networks, environmental distance and patterns of migration.
\\Ecological events eliminate barriers and allow species to migrate into new environments, introducing new sets of genes in an environment. The presence of variant genes and network structures from gene flow hinders local adaptation and fixation of adaptive genes \cite{burt1995evolution}. Using quantitative trait loci (QTL) we are able to numerically interpret and visualise the patterns of change. Previous research looked at the effects of gene flow, selection and mutation at generating local adaptation at the phenotypic level, showing how maintenance of alleles and linkage is important in adaptation \cite{yeaman2011genetic}. It was shown that with random perturbations and aid of genetic modifiers, there are bounds for which selection for canalization can act on, leading to evolution of robustness. They also showed that under migration selection balance, selection for robustness increases with the migration rates \cite{proulx2005opportunity}. This research will be looking at the changes in genetic architecture dynamics and the interactions of the varying systems. As a genetic network evolves, there exists a threshold which is actively regulating these homeostatic genes \cite{gjuvsland2007threshold}. As selection for robustness occurs within the local population, it can give insight into the change in architecture and statistically significant interaction \cite{gjuvsland2007statistical}.
\\Using a multi-locus system, I will construct a genetic network and simulate the effects over many generations and see how the output of the network changes, specifically looking at allelic interactions and tracking the fitness over time. Variance in fitness should decrease as a genetic network becomes robust, making it resistant to perturbations. Fitness can be quantified as reproductive success and is represented by passing on quantitative values generated from the alleles. These values are used to derive the trait values of individuals of which phenotypic values are then calculated and used as probabilities for fitness. The expectation is that after migration, a more robust network is formed when compared to before migration. At the start allowing new alleles to enter the population will result in a less robust network and more susceptible to perturbations \cite{garcia1997genetic}. Especially when the migrant network is a different structure, gene flow will allow maladaptive alleles to enter and should those be passed on, will impose a fitness cost to individuals \cite{tigano2016genomics} However, over time, the network should adapt and become resistant to such perturbations.
