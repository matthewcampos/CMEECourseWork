% ITEX root = ../Campos_Matthew_CMEE_2020.tex
\subsection{No Migration}
Previous studies showed how a reproductively isolated population would evolve rapidly \cite{garcia1997genetic}. Without gene flow, populations did evolve rapidly to their environment due to a balance between selection and drift \cite{garcia1997genetic,tigano2016genomics,barber1999patterns}. Especially for a heterogenous genetic network, which would evolve a lot quicker which was the case. Compared to a population that started as homogenous, heterozygous individuals had variation in allele values at the start allowing for better combinations of values for selection to act on. This allowed such values to propagate and spread in the population. Swift rise (within 100 generations) can be attributed to the fitness distribution and the non-limitations in reproduction of the simulation. Following the fitness distribution, the most fit individuals per generation were likely to have been selected for reproduction many times allowing their combination of alleles to be passed on at a high frequency. Thus, the slopes of the heterogenous plots in \textit{figure 3} are steep, suggesting selection was strongly acting on the population. Deviations seen are likely due to mutation and recombination events that removed favourable allele values and combinations.\\
In the homogenous case, the instances where fitness rose close to 1 is likely due to chance beneficial mutations and recombination to create better combination of allele values, and epistasis which drive the adaptiveness \cite{tigano2016genomics}. A beneficial mutation in $gene_1$ for example can have positively consequences for the values in $gene_2$ and $gene_3$. With selection and non-limiting reproduction acting on the system, it allows for such values quickly spread in short time, raising the fitness of the population. Additionally, \textit{figure 2} shows the magnitude in difference of deviating towards from the median of the distribution. For the desired trait value of 50 with a standard deviation of 8, the maximum probability an individual has of being chosen is 3.98\%. Although this is a small value, an individual that is 10 trait values away will have a probability of 1.55\%, a three-fold difference between them. So, selection and limitless parental contribution allows even a small number of fitter individuals to be parents many times and pass on their values.
\subsection{Migration and Robustness}
To begin with genetic regulatory pathway of a population is seen to effect recovery time. Boxplot in \textit{figure 5} shows that recovery time for a migrant population with a different regulatory network, one that evolves to a trait value of 65, and same regulatory system evolving to trait value 80 are quickly diminished from the population. However, a different migratory network that evolves to 80 or the same migratory network evolving to 65 persists in the population longer. Other than beneficial mutation or recombination events, there could be a possibility that gene flow helped to improve fitness. That patterns caused by evolutionary forces could be context-dependent. For a homogenous population that evolves slowly in, gene flow could lead to variant alleles closer to the peak of the distribution, hence a greater number of rising peaks in \textit{figure 4b and 4c}. Further runs and tests must be conducted to statistically observe if gene flow has a beneficial effect depending on environmental distance and genetic makeup.\\
For heterogenous populations, \textit{figure 4} shows how gene flow has an expected negative effect on fitness. It leads to large deviations in the fitness. Poorly adaptive traits to the environment would delay the system ability to evolve to local optimum as the values deviate far from it \cite{garcia1997genetic}. This was to be expected as periods where the foreign alleles far from the peak of the distribution. trait value and fitness to deviate from the local optimum.\\
Analysis of variance test revealed statistical significance of migration pattern on the log ratio of robustness. Although post-hoc Tukey test revealed no significance within the combinations of migration pattern, this is likely due to small sample sizes. Further testing will likely reveal significant patterns between migration rate and canalisation of a network. There was still great insight on how the genetic network was able to overcome disruptions caused by gene flow and prevent homogenizing the populations and resisting perturbations. While gene flow caused large deviations in the fitness (as seen in \textit{figure 4}), selection counteracts gene flow and maintains the adaptiveness of the network \cite{feder2012genomics,burt1995evolution}. Moreover, I could investigate the patterns at which migration, selection and drift affect population growth, analysing effect of dominance on robustness \cite{rice2009evolution,otto1999balanced}. When migrant alleles enter the population, recombination gives them the opportunity to persist \cite{feder2012genomics}. Not just migrant allele values, but also complementary combination of migrant alleles that could allow it to persist \cite{feder2012genomics}. Yet despite the presence of migrant alleles, divergent selection heavily favours local alleles in particular and is quick to remove them \cite{tigano2016genomics}. Selection favouring locally fit individuals with better combination of alleles. \textit{Feder (2012)} study highlighted a friction that exists between divergent selection, and gene flow and recombination. Results in this study show how when selection force is greater, it acts to swiftly remove migrant alleles in the population.\\
\subsection{Shortcomings and Future Work}
Due to time constraints and difficult circumstances, major assumptions were made in the model and simulation. For example, the fitness distribution in \textit{figure 2} shows how variation was quickly eliminated from the model due to the narrow shape of centre of the distribution, allowing for selection to act strongly on the network. Few trials as well of the different conditions and design of the simulation prevented significant results from appearing. Realistically, the model did not take into account spatial or temporal aspects. Species are distributed along a habitat and the environment itself can change due to perturbations i.e. vicariant events \cite{garcia1997genetic}. With more time, I would have designed a more realistic model, taking into account dispersal habits, species distribution, environment perturbations, and migration limits. Species dispersal along with varying environmental conditions would have created a more complex system and delayed the quick homogenization of the network to allow for investigation of gene flow effects on fitness \cite{garcia1997genetic,barber1999patterns,sato2006effect}. This also would better maintain the variation in the population to better analyse the impacts. This would prevent the population being completely replaced in the current model even with higher migration rates. With regards to environmental spatial dynamics, I would have made it so that different areas would have different local optimums.\\
The current simulation assumed that there is no limit to parental contribution. Since there was no limit, the fitter individuals quickly pass on their trait values in the simulation. This rapidly canalised the system and decreased the variation in the environment. To improve this, I would limit parental contribution i.e. individuals reproducing can only be parents to five next generation individuals. This would prevent the system from stabilising so rapidly and allowing for more diverse alleles to continue to propagate in the population. There were no ecological barrier restricting migration and individuals were always replaced by migrants. Random environmental perturbations that would alter local optimums would further delay network evolution allowing for longer periods of variance. Thus, the effects of selection would not so quickly evolve the system and can better investigate how network responds to disturbances. This would create more varying network structures and allow further investigation into network development with varying motifs.\\
Overall, further runs must be conducted to see effects of migration pattern on robustness. Although the post-hoc Tukey test showed no significance, it likely due to small sample size of runs per starting condition \textit{(see Appendix A)}. In addition, another aspect to research is the possible benefits of gene flow on an evolving network. More runs can give statistical insight as to whether this is the case, by tracking lineages and allele values. Moreover, I can also consider dominance and investigate how the migration, selection and drift affect population growth, analysing effect of dominance on robustness \cite{rice2009evolution,otto1999balanced}. Lastly, it would also be interesting to investigate the opposite case as to how migration patterns are influenced by these evolutionary forces \cite{w2004dispersal}.
