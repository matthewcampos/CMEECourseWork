\documentclass[11pt]{article}
\usepackage{graphicx}
\usepackage{amsmath}
\usepackage[utf8]{inputenc}
\usepackage{setspace}
\usepackage{caption}
\usepackage{subcaption}
\usepackage{wrapfig}
\usepackage{caption}
\captionsetup[figure]{font=footnotesize}
\usepackage{lineno}
\usepackage{harvard}
\usepackage{parskip}
\usepackage{indentfirst}
\usepackage[a4paper,width=150mm,top=20mm,bottom=20mm,bindingoffset=6mm]{geometry}


\title{CMEE Seminar Diary 2019-2020}
\author{Matthew Campos\\MSc. CMEE\\Faculty of Natural Sciences}

\date{2nd July 2020}

\begin{document}

    \maketitle

\newpage

\section{Deep-time evolution of biological responses to temperature changes}
\textit{Dimitrios-Georgios Kontopoulos\\Imperial College London\\10/10/2019}
\\
\\Species are adapted to their respective environments and with a changing climate, the research of Georgios Kontopolous et. al is to investigate how physiology, through enzyme activity would be impacted by changing levels of heat. The experiments carried out were to view it from different levels of biological organisations from individual, to population and finally species to species interaction level. The research focused on the processes that lead to variance in evolutionary adaptation to heat tolerance and if these traits are similar among related species. Thermal Performance Curves and variant of the Sharpe-Schoolfield model would be used to understand how physiological processes of phytoplankton and bacteria, and mutation of the protein Adenylate kinase would be impacted by varying temperatures. Results show that thermodynamic constraints have little influence on the curves and that species adaptation is not fixed on these thermodynamic constraints. Traits evolve depending on the environment they are in, which is affected by latitude. Finally, these traits are influenced by enzyme activity and high temperatures increases the likelihood of detrimental mutations. Thus, species living in hotter environments tend to evolve slower to decrease the probability of a destabilising mutation occurring.

\section{A manifesto for systematically describing consumer-resource interactions}
\textit{Daniel Barrios-O’Neill\\University of Exeter\\31/10/2019}
\\
\\Barrios-O’Neill research focuses on consumer-resource interaction, specifically looking at predator-prey interactions. Throughout the years, many experiments have led to understanding the functional responses in nature, and the influence of factors including body size and temperature. However, Barrios-O’Neill argues to take a more systematic approach of understanding the unexplained patterns. Rather than trying to reproduce works from the past, explore patterns that are yet to be understood, through experimentation utilising new and current technology. Overall, he takes a holistic approach in understanding the three types of functional responses. He explains how interactions are context dependent, including the taxon of the organism, how the consumer behaves, actively or passively searching for resource and the type of habitat structure where the interaction occurs. His results found that functional curves produced depend on factors including space and handling time. He investigates many aspects that affect functional response ranging from physiological to spatial, how these complexities of habitats and behaviours and physical constraints of the organisms influence the interaction. Looking at encounter strategies and parsing habitats structures, he creates artificial spatial scales and carefully manipulating the conditions to see how the responses change. He regards that accuracy of data is limited and skewed by the model system and species.

\section{Finer-scale and standardised mosquito surveillance provides a new path to predict dengue outbreak risks}
\textit{Dr. Ruiyun Li\\Imperial College London\\14/11/2019}
\\
\\Dengue is a climate and seasonal influenced virus transmitted by certain species of mosquitoes, prevalent in the tropics and subtropics. Dr. Li's research collected data in China, where there has been three major dengue outbreaks with the virus spreading from coastal regions to inner provinces. Recently, there has been increasing incidence rates of dengue in these areas. Li et. al research was conducted in 5 cities in China, investigating the mechanisms of climate-epidemic interactions. The research revealed that the local predictions of risks and transmissions differs of that in a global scale. The purpose of the research was to combine mechanical and statistical models to predict the rate of human incidence, utilising an SIR model. The mechanical aspect was the predicted mosquito abundance based on the data collected in the three regions, and the statistical model, the climate conditions. Mosquitoes were captured during the summertime in three divided regions in China along with weather data regarding mean temperature and number of precipitating days. Using the data, it would be possible to predict incidence rates, using mosquito vector efficiency. Transmission rate is the product of per mosquito vector efficiency, and the estimated mosquito abundance. Although findings differed per region, and was limited by surveillance and control, overall trend shows that the model matched the observations accurately, with higher precipitation during summer leading to higher abundance rates.

\section{Flowers, bees and shifting seasons- how to adapt when Nature’s calendar goes out of sync in a warming world}
\textit{Jacob Johansson\\Lund University\\21/11/2019}
\\
\\Phenological mismatch is the research focus of Jacob Johansson. Due to climate change, it has caused a variation in the rates of change among species and events. As a result, this asynchrony has led to species growth at suboptimal times. This negative fitness effect leads to adaptive mechanisms to track seasonal optima, which leads to mixed responses as certain processes benefits, while others are harmed. Johansson et al. studied showed the species have different responses to this phenological mismatch. Specifically looking at flowering times and the effect of mismatch of flowering times with pollinators such as bees. Investigating \textit{Clarkia rubicunda} and \textit{Bombus terrestris}, the research investigated the production rate of plants and how it changes when conditions are manipulated, and the factors that lead to the changes. Plants have a process of dynamic energy allocation where they begin in a vegetative part, increasing production, before switching to the reproductive phase, where the switch time is determined by the production rate. Adaptive phenological responses vary and population trends vary, depending on life history trade-offs and competitive effects. Two strategies are noticed, species either shift or don’t shift. Some species adapt by finding a new optimal time with minimal trade-off between reproduction and resource availability. However, some species choose not to shift and maintain their seasonal processes. Adaptive phenological responses and population responses vary as a result from phenological mismatches. In the short term, population is expected to decline however competitive release can cause population increase. Interspecific competition between species leads to uneven shifts in season resource distribution, leading to asymmetry in population, reinforced by adaptation. Some species also decide not to shift and exploit the increased resource availability while competitors are reproducing earlier.

\section{The complex consequences of simple sociality in the wild}
\textit{Josh Firth\\University of Oxford\\05/12/2019}
\\
\\Firth et al investigates the formation of social behaviour which includes competition, cooperation and disease spread, and the cost benefit of interactions. Investigating the Wytham Tits in Wytham woods, the study focused on the importance of social bonds and social phenotypes. Using data permutations or null models, they were able to generate null distribution of random behaviour and see where significant observation statistic lies. Taking into account spatial distribution and social structures, the birds were tagged with RFID to control foraging flocks, different experiments were conducted to see how social bonds and association between birds change. The purpose was to see the degree of social bonds and if bond strength would maintain social behaviour. Correlation or causation was investigated by manipulating social interaction by choosing a selective feeder which only feeds birds of certain RFID. Repeatability measurement showed consistent social traits. These social bonds are important as well for reproduction and to see how associations are formed and influenced, leading to territorial formations and spatial structure. Experiments explored how social behaviour of individuals changes by removing birds, affecting social connections. The would be able to predict how close birds will breed together based on how much time spent together and looking at arrangement of territories so distribution of territories within area through social bonds. The study shows that neighbours have strong bonds as oppose to non-neighbours. Results showed complex social structure and explained the social phenotypes. Furthermore, that the greater the number of flock mates removed, the more active they become in finding new flock mates and replenish and increase the strength of the network. With regards to complex and simple interactions, the network position of an individual and number of interaction correlates to problem solving abilities. This is affected by how well an individual bridges two different groups as well as the transitivity of the group. Solvers tend to have many connections, to be able to spread simple information faster, but degree of connections are weaker compared to non-solvers even with accounting for spatial structure. Further research is going on to investigate complex contagions, which take into account degree and strength of connections.

\section{Effects of temperature on Microbial Metabolic Rates: Linking Individual Responses to Ecosystem Impacts}
\textit{Tom Smith\\Imperial College\\23/01/2020}
\\
\\Smith et al, investigated the effects of temperature on microbes, looking at bacteria. Particularly the direct effects on metabolism and growth rates with varying temperatures. With the impacts of climate change, this will have a direct effect on metabolic and growth rates from the global temperature increase. Smith was concerned if bacteria obey similar rules to thermal sensitivity. Across species, mass-specific metabolic rate seems to show a similar thermal sensitivity trend as within species, showing that there is a universal temperature dependence which is governed by the same thermal constraints. In addition to the experiment, he also compared results to published literature. Environments with higher proportions of microbes to other organisms tend to have higher thermal sensitivity and the data reveals short-term “instantaneous” thermal response. Smith et al, were curious if there was a link between short term and long term adaptation. Using the Schooling-Sharpe model and analyzing thermal performance curves, results show short-term instantaneous thermal response, a sign of adaptation. Microbes adapt by shifting peak so metabolic rate will increase with increasing temperature. In addition to this, since microbial communities are known to have high diversity, gradient of peak of temperatures relate to species sorting processes, selecting for aspects already existing in soil best adapted for temperature changes. Increasing temperature benefits the bacteria known to have specialized growth mechanisms. Peak temperatures are found to be influenced by phylogeny. Specifically, those with high temperature peaks are growth specialists but have low carrying capacity. Those with lower temperature peaks have low growth rates but are more efficient with resources having higher carrying capacities. With increasing temperatures, the prediction is that selection for growth specialists will be favoured.

\section{Phylogenetic signature of interspecific competition in birds}
\textit{Jonathan Drury\\Durham University\\30/01/2020}
\\
\\Drury et al, investigated interspecific competition in birds, specifically passerines, and understand adaptive radiations, focusing on related species in evolving clades. The purpose was to understand species interactions between trophic levels to understand ecosystem dynamics and adaptive radiations. Using phylogenetic comparative methods, they assessed character displacement to see how competitive interactions drives evolution of traits. Competitive interaction drives evolution of traits which leads to character displacement as selection favours divergence. Three projects were set up looking at interspecific territoriality, competition in large continental scales to see which traits are impacted and latitudinal gradients in competition. Studying passerines, the first project focused on intraspecies interactions. Aggression between species maintain ranges and this is caused by syntropy, songs and plumage. Results were consistent with adaptive hypothesis as resources competition and mate competition drove territoriality. Species that are ecologically similar are more likely to have infraspecific territoriality and show aggression towards each other. The second project looked at trait divergence that are sympatric or allopatric. Using a process-based approach, they focused on the tips of phylogeny and traits and used different models to produce trait dataset specifically of tanagers. Dataset produced included resource use traits and social signalling traits. Rate of evolution and morphological disparity index show song and plumage rates evolving match faster compared to resource as similar species are also likely to be allopatric with sexual selection driving the difference. Finally, the final project looked at latitudinal competition and interactions between trophic levels. Using morphological data and life history, they looked at competition between members of an evolving clade promoting divergence, family level analyses, and competition between members of different lineages restricting diverging, continental level analyses. Results suggest that there was no evidence of tropic influence, but latitudinal gradient shows rapid diversification of traits in certain cases.

\section{Predicting Global Biodiversity with General Ecosystems Models and Neutral Theory}
\textit{Lucas Dias Fernandes\\Imperial College London\\04/02/2020}
\\
\\Fernandes’s research is predicting global biodiversity by incorporating both General Ecosystem Models and Neutral Theory. Patterns of biodiversity can be explained in many layers including phylogenetic, taxonomic, functional and their interactions and the research purpose is to understand the complexity and effects of anthropogenic change to predict the effects on biodiversity. Mechanistic models are to understand complexity as they are built from constituent principles to see how the parts and units interact by scaling up to interaction in large habitats. Using these models, causal effects may be investigated understanding emergent patterns. The General Ecosystem Model used is a Madingley Ecosystem model which simulates individuals categorised by different functional groups, taking input of environmental and data and simulating the interactions to visualize the patterns and effects of seasonality on ecosystems. Taking into account demographic rates and simulate life history traits, it generates a realistic ecological pattern. Although it can generate ecological patterns on a global scale, it does not extend to species-level indicators. For this, Neutral theory is the mechanistic model used to understand other aspects of biodiversity as it focuses on species variation over time. Combining the models allows using abundances, dispersal, reproduction, maturity, and death rates from Madingley, and species identities and spatial distributions from neutral theory. Using these spatial distribution and species identities, it becomes possible to track species variation over time and observe patterns of relative species abundance and species area relationships. Therefore, becoming better at tracking life history traits and include ecological mechanisms. Despite the limitations of predictive models, the integration of both Madingley model and Neutral Theory provides species level description for biodiversity patterns.

\section{Sex and Drugs and Bee Control: paradox of plant toxins in nectar}
\textit{Philip C Stevenson\\University of Greenwich\\06/02/2020}
\\
\\Ecological interactions and evolution lead to some plants being able to produce toxins in nectar. These chemicals affect behaviour of insects for agriculture and animal health. Stevenson et al is concerned with understanding their ecological functions and chemistry can help with land management and develop solutions for pollination challenges and maintain diversity and health. Furthermore, understand interactions of pollinators and plants. The toxins which have many ecological purposes including defense mechanism, are known to affect behavior of insects which has ecological and physiological consequences. Nectar compounds inform our understanding of plant pollinator interactions. For example, \textit{Lupinus mutabilis} produces the toxin D-lupanine which can be found in bee pollen leading to fewer and smaller offspring. While other toxins such as callunene has antimicrobial effects and prevent diseases in bees. Plant toxin levels vary spatially across native and introduced species and is correlated with fitness. For introduced species, nectar production hints at adaptation by filtering the possible pollinators. Another function is to prevent nectar robbery and ensure only specialized species retrieve the nectar. Aconiutum nectar contains deterrent and entomotoxic alkaloids and concentration determines if visited by robber or pollinator. The toxins are also known to manipulate pollinator behavior. Caffeine for example is a defense mechanism however enhances excitability and sensation of bees and make them better pollinators as they are more decisive and efficient at foraging time as it depolarizes membrane potential of Kenyon cells which integrate stimuli during associative learning and memory formation, enhances excitability and sensation.

\section{Evolutionary causes and consequences of avian dispersal syndromes: importance of individual variation in colonisation processes}
\textit{Marion Nicolaus\\University of Groningen\\27/02/2020}
\\
\\Nicolaus et al, research focuses on avian dispersal syndrome and the evolutionary causes and consequences of it. Species disperse from habitats for multiple purposes including resource or habitat colonization with ecological and evolutionary impacts. Dispersal syndrome relates to a non-random subset of a population influenced by morphology and physiology dispersing, and with changing habitats leads to ecological and evolutionary processes such as non-random gene flow, affecting population genetic dynamics. Thus, a difference between dispersers and non-dispersers. For example, overtime phenotype of aggressiveness replaced by non-aggressive philopatric in Western Blue birds as it becomes a maladaptive trait. The three possible hypotheses to explain such phenomenon are that dispersers are non-random as immigrants differ in heritable traits that facilitate settlement in new environments, population trends in newly established populations changes genetically in dispersal and underlying mechanism leading to variation in frequency of types caused by change in correlational selection. Looking at \textit{Ficedula hypoleuca}, the research conducted a whole organism approach to get a better understanding of how evolution proceeds and compared dispersers and non-dispersers to see if there was dispersal syndrome happening and if correlational selection was present. Results showed there was an existence of disperser syndrome as immigrants had certain characteristics that differentiated them from philopatric individuals Including being larger and higher fecundity. Quantitative genetics showed that these traits driven by individual effects are highly heritable which has the potential to evolve. However, there was no sign of correlational selection but rather direct and indirect selection during colonization. There was a positive association between competition and aggression, which was supported by adaptive plasticity. Aggression-dispersal covariation is underpinned by adaptive plasticity. Dispersal causes the translocated birds to exhibit higher level of aggression than control, plasticity, as individuals decrease levels of aggression over time. However, results showed that depending on the years certain phenotypes are favoured. This supports general pattern that fluctuating selection maintains variation in the wild.


\end{document}
