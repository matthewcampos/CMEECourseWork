
\section{Frogs}

        During your latest field trip in Costa Rica you observed how brightly coloured
        poison dart frogs (part of the \textit{Dendrobatidae} family) were.
        In fact, the brightness of their skin colouration is correlated with their toxicity.
        To investigate the prevalence of toxic frogs in the area under study, you collected
        $n$ samples of poison dart frogs and observed that $k$ of them have bright skin colour
        (and thus are toxic).

        \paragraph{Question A}

        You now want to estimate the \textbf{population} frequency of the red colour
        phenotype, $f \in [0,1]$.
        Assuming a generic likelihood function $p(k|f,n)$, where $k$ is our observed data,
        and prior distribution $p(f)$, write the expression for the posterior distribution
        of $f$, $p(f | n,k)$.
        Indicate the interval for the integration over $f$ explicitly.
        Finally, if $p(f|\mu)$ (assuming we have uncertainty on the prior for $f$
        with parameter $\mu$, what is the resulting full equation for the posterior
        distribution $p(f | n,k,\mu)$?

        \paragraph{Question B}

	If we define the likelihood function as a Binomial distribution:
        \begin{equation}
                p(k|f,n) = ( \genfrac{}{}{0pt}{}{n}{k} ) f^k(1-f)^{n-k}
        \end{equation}
        and the prior function as a Beta distribution $B(\alpha,\beta)$:
        \begin{equation}
                p(f) = \frac{1}{B(\alpha,\beta)} f^{\alpha-1}(1-f)^{\beta-1}
        \end{equation}
        then the posterior distribution of $f$ is a Beta distribution with
        parameters $\alpha'=k+\alpha$ and $\beta'=n-k+\beta$.

        What is the frequentist estimate of $f$?
        What is the maximum likelihood estimate of $f$?
        What is the maximum \textit{a posteriori} mode of $f$ using the noninformative
        conjugate prior $p(f) \sim B(\alpha=1,\beta=1)$?
        What is the maximum \textit{a posteriori} mode of $f$ assuming $k=12$ and $n=235$ using the
        uniform prior $p(f) \sim U(0.5,1)$?

        \paragraph{Question C}

        Assuming we collected 100 samples and 25 of them have bright skin colour,
        please complete the \texttt{R} code below in order to generate both the exact
        and approximated posterior distribution of $f$ using the informative
        prior $p(f) \sim B(\alpha=2,\beta=1)$.
        You need to fill in where the string '???' is present.
        Finally, discuss which summaries you would report to describe the resulting posterior
        distribution.
	\begin{lstlisting}[language=R]
                # we evaluate our parameter f over a grid of 100 values for the whole range [0,1]
                f <- seq(0, 1, ???)

                # suppose we collected 100 samples and 25 of them have bright skin colour
                k <- 25
                n <- 100
                # alpha and beta are the parameter values for the posterior Beta distribution
                alpha <- ???
                beta <- ???

                # we now evaluate the density function to obtain the EXACT posterior distribution
                y <- ???(???, shape1=alpha, shape2=beta)

                # we now use Monte Carlo sampling to obtain the APPROXIMATED posterior distribution.
                Make a reasonable choice for the number of random samples.
                y_sampled <- ???
                y_sampled_distribution <- ???

        \end{lstlisting}

        \paragraph{Question D}

	If we use a Normal distribution as prior information, such as
        $p(f)=N(\mu,\sigma^2)$, we cannot derive a closed form and cannot sample
        directly from the posterior distribution.
        We can use a rejection sampling algorithm for \textit{indirect} sampling of the
        posterior distribution.
        This algorithm requires the identification of an envelope function $g(f)$ and a
        constant $M>0$ such that $p(k|f,n)p(f)<Mg(f)$.

        Identify both a suitable envelope function $g(f)$ and a value for $M$ assuming that
        we know that the maximum density value for the posterior distribution is $K$.
        Describe the full algorithm (with formal notation for equarions, e.g. $\nu \sim P(\lambda=2)$)
        or write its pseudocode (as accurate as possible).
        Describe what happens to the algorithm and/or the approximation if we
        choose $M>>K$ or $M<<K$.

        If we want to use an MCMC sampling approach, is it possible to use a Gibbs sampler in this case
        or is it more appropriate to use a rejection algorithm like the Metropolis one? Why?


