% ITEX root = ../Thesis.tex
In conclusion, the Gompertz model performed well due to its derivations. As mentioned in the Discussion, the parameters and its values are empirically derived. Similar to the Logistic model, its parameters do not consider the physiology or limiting factors of the organism, just the rate at which it divides. Furthermore, compared to the Logistic model, it takes in an extra parameters that makes it more robust and flexible.

When fitting models, there is a tradeoff between realism and complexity. Although complex models, which contain more parameters, will fit the given data set better. This is however, all dependent on the context and purpose of study. In this research, the focus was on  models that can explain the patterns and trends in the data, and would fit well relative to the others. Only two models, Baranyi model and Buchanan model actually considered organismal physiology and further model fitting can be done using other biologically mechanistic models that take into account internal and external factors.

For the additional investigation, although the distribution of the data did not match that from the study carried out by \textit{Ratkowsky et al.}, and only used five species, the overall result still supports the claim that growth rate is positively correlated to increases in temperature. However, the derived conceptual temperature from the linear regression did not match that provided from the paper. Replication and additional investigation can be carried out to see if increased dataset leads to this condition, and if the positive correlation hold true for the other microbes.
