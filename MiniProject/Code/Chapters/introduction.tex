% ITEX root = ../Thesis.tex
In the field of Biology, mathematical models can be used to explain or describe natural phenomena and observations. This method of modelling can be applied especially in the field of Ecology and Evolution. Models are a simplistic method to describe the general behaviour of long-term macro-biological events noticed in nature \cite{johnson2004model}. A field in Ecology is Population Ecology, and simplified models can validate the dynamics of population growth and how different factors and conditions affect the population numbers over time. The models chosen are generalised models describing bacterial growth. Therefore, they disregard the capacity for evolution, classifying the organisms as homogenous \cite{levins1966strategy}. Finding the best fit will be achieved by fitting suitable models and evaluating the best fit to the data using either maximum likelihood or least squares \cite{johnson2004model}. Model fitting has become an alternative approach to the traditional null hypothesis tests when carrying out scientific experimentation. If carried out properly, parameters and overall function of the model are used to make biological inferences on the patterns recorded \cite{johnson2004model}. The challenge, however, is finding the appropriate model and taking into considerations the assumptions and trade-offs of the different models.

Models aim explain the important aspects of the data, thus have many assumptions. When choosing a model, it is important to account for the trade-offs between generality, realism and precision, focusing on model that sustain generality and realism, as they are able to quantitatively explain long-term trends that the data reveals \cite{levins1966strategy}. In other words, the purpose of the model, and what is being investigated is an important factor. In addition to this, ensuring that the parameters have biological significance rather than just statistical significance \cite{johnson2004model}. For this project, identifying the correct model is achieved through the process of testing multiple existing mathematical and biology models on the data and finding the model that best fits the data using the analytical method of Maximum Likelihood. The outcome should be plausible models that for most observed data, consistently describes the patterns in these observations mechanistically or empirically.

The focus of this project is to test multiple non-linear regression models on microbial population data. Identifying the best model(s) that best fit and or describe the pattern of the data. Microbes grow through a process known as binary fission, doubling the population continually \cite{webb1986logistic}. This exponential growth coupled with limiting factors gives microbial population growth a sigmoidal shape which can be divided into four phases \cite{peleg2011microbial}. The trend begins with birth, a starting population (N0) and growth where limiting factors is not yet inhibiting, thus having an exponential growth curve, with a growth rate notation of r \cite{zwietering1990modeling}. Limiting factors such as space, nutrients and particularly carrying capacity can inhibit growth, resulting in the rate of growth to decrease and eventually reaches saturation, becoming asymptotic. The asymptotic value being the carrying capacity, with the notation $N_max$ \cite{zwietering1990modeling}.

Four mechanistic models will be tested on the data set. These are the Logistic Growth Model \cite{bacaer2011verhulst}, Baranyi Model (Baranyi,1993), Gompertz Model \cite{zwietering1990modeling} and Buchanan Model \cite{buchanan1997simple}, 1997), given below.

\textit{Logistic Model}
\begin{equation*}
    N_t = \frac{N_0 \cdot N_{max} \cdot e^{r_{max} \cdot t}}{N_{max} + N_0(e^{r_{max} \cdot t} -1)} \label{eq:Logistic Model} \tag{1.1}
\end{equation*}

\textit{Baryani Model}
\begin{equation*}
    N_t = N_{max} + \log_{10}(\frac{-1 + \exp(r_{max} \cdot t_{lag}) + \exp(r_{max} \cdot t)}{\exp(r_{max} \cdot t) - 1 + \exp(r_{max} \cdot t_{lag}) \cdot 10^(N_{max} - N_0)}) \label{eq:Baranyi Model} \tag{1.2}
\end{equation*}

\textit{Gompertz Model}
\begin{equation*}
    N_t =  N_0 + (N_{max} - N_0) \cdot \frac{\exp(-\exp(r_{max} \cdot \exp(1) \cdot (t_{lag} - t)}{(N_{max} - N_0) \cdot \ln(10) + 1})) \label{eq:Gompertz Model} \tag{1.3}
\end{equation*}

\textit{Buchanan Model}
\begin{equation*}
    N_t = \begin{cases}
          N_0, & \text{if } t\leq t_{lag}\\
          N_{max} + r_{max} \cdot (t - t_{lag}), & \text{if } t_{lag} < t < t_{max}\\
          N_{max}, & \text{if } t\geq t_{max}
          \end{cases} \label{eq:Buchanan Model} \tag{1.4}
\end{equation*}
Model fits be will be assessed using the Akaike Information Criterion (AIC), which is a relative comparison where the best fit is the model with the minimum value \cite{vrieze2012model, posada2004model}. The AIC is a criterion used to assess the model fits. It is a relative measure of the Kullback-Leibler divergence estimate. Meaning that whereas the K-L divergence measures the deviation of a model from the "true model", the AIC is an estimate value only between the models considered \cite{vrieze2012model}. It consists of two variables, the likelihood function (a loss function) which results in a Maximum Likelihood Estimate of the parameters, and the number of parameters in the model \cite{vrieze2012model}. The formual is given below, where $\alpha$ is known as the penalty coefficient (with a value of 2), $\kappa$ is the number of parameters and $\mathcal{L}$ is the Maximum Likelihood Estimate of the parameters.

\textit{Akaike Information Criterion}
\begin{equation*}
    AIC = -2 \cdot \ln{(\mathcal{L})} + \alpha \cdot \kappa \label{eq:AIC} \tag{2}
\end{equation*}
Lastly, an additional investigation on the correlation between growth rate and temperature is also carried out. In addition to this, compare two models, a linear model and a relationship formula given by \textit{Ratkowsky et al.} Given the range of temperature from 0 degrees Celsius to 37 degrees Celsius, the expectation is to see a positive linear relationship between increasing temperature and growth rate. Species and medium are factors considered and a linear regression anlaysis is done, as well as plotting the formula from \textit{Ratkowsky et al.} given below.

\begin{equation*}
    \sqrt{r}=b(T-T_0) \label{eq:Conceptual Temperature model} \tag{3}
\end{equation*}
I will call this formula the Conceptual Temperature model. The parameter $T_0$ is the conceptual temperature, a hypothetical temperature which has no metabolic significance, in which other factors such as medium are non-limiting \cite{ratkowsky1982relationship} Conceptual temperatures for different species are provided from \textit{Ratkowsky et al.} paper and used to identify the relationship between growth rate and temperature and see if the model performs better than a linear regression.
